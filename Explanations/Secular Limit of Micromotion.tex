\documentclass{article}
\usepackage{graphicx}

\usepackage{amsmath}
\usepackage{amssymb}
\usepackage{enumerate}
\usepackage[utf8]{inputenc}
\usepackage{graphicx}
\usepackage{caption}
\usepackage{subcaption}
\usepackage{hyperref}
\usepackage{placeins}
\usepackage[title]{appendix}
\usepackage{float}
\usepackage{comment}
\usepackage{multicol}
\setlength\columnsep{10pt}
\usepackage{soul}
\usepackage{gensymb}
\usepackage{enumitem}
\usepackage{subcaption}
\usepackage{physics}
\usepackage{color}

\usepackage{fancyhdr}
\usepackage{currfile}
\usepackage[us,12hr]{datetime}
\fancypagestyle{plain}{
\fancyhf{}
\renewcommand{\headrulewidth}{0pt}}

\usepackage[margin=0.6in]{geometry}

\title{Secular Limit of Micromotion}

\begin{document}

\maketitle
\noindent The motion of two ions in a linear Paul trap can be generated by solving the following Mathieu Equation, derived previously:
\begin{equation}
v_p''(\tau) = - \frac{(2 \pi)^2}{\beta^2} \left( a_p - 2 q_x \cos \left( \frac{4 \pi \tau}{\beta} \right) \right) v_p (\tau)	
\label{eq:mathieu}
\end{equation}
If we make the substitution $\xi = \frac{2 \pi \tau}{\beta}$ (which is equivalent to $\xi = \frac{\omega_{RF}}{2} t$, as $\tau = \frac{\beta \omega_{RF}}{4 \pi} t$), then the Mathieu Equation can be expressed more simply as:
\begin{equation}
v_p''(\xi)	= - \left( a_p - 2 q_x \cos \left( 2 \xi \right) \right) v_p (\xi)
\label{eq:mathieusimple}
\end{equation}
The Mathieu Equation is one of many differential equations with periodic coefficients, the stable solutions of which can be found from Floquet theorem:
\begin{equation}
	v_p (\xi) = A e^{i \beta \xi} \sum_{n = - \infty}^{\infty} C_{2n} e^{i 2 n \xi} + B e^{- i \beta \xi} \sum_{n = - \infty}^{\infty} C_{2n} e^{- i 2 n \xi}
	\label{eq:motion}
\end{equation}
The contribution from the oscillating component of the trap is included in the summations -- that is, the summations capture all contributions to the motion from the micromotion. The exponential terms out the front of the summations will determine the secular motion of the ions. 
%%%%%%%%%%%%%%%%%%%%%%%%%
% How was this determined? 
If $\xi = \frac{\omega_{RF}}{2} t$, then we can see that the dimensional frequency of the secular motion (ie. in units of inverse dimensional time, $1/t$) will be given by:
\begin{equation}
\omega_{\text{sec}} = \frac{\beta \omega_{RF}}{2} \label{eq:sec}
\end{equation}
By inserting the motion in \eqref{eq:motion} back into \eqref{eq:mathieusimple}, one can obtain recurrence relations for the coefficients $C_{2n}$. With this substitution, the LHS of \eqref{eq:mathieusimple} becomes:
\begin{equation*}
v_p'' (\xi) = - \sum_{n = - \infty}^{\infty} C_{2n} (\beta + 2n)^2 (A e^{i (\beta + 2n) \xi} + B e^{-i (\beta + 2n) \xi} )
\end{equation*}
Accordingly, the RHS becomes:
\begin{equation}
- \left( a_p - 2 q_x \cos \left( 2 \xi \right) \right) v_p (\xi) = - \sum_{n = - \infty}^{\infty} C_{2n} ( a_p - 2 q_x \cos \left( 2 \xi \right) ) (A e^{i (\beta + 2n) \xi} + B e^{-i (\beta + 2n) \xi} )
\end{equation}
Equating the LHS and the RHS, we achieve: 
\begin{gather*}
	- \sum_{n = - \infty}^{\infty} C_{2n} ( a_p - 2 q_x \cos \left( 2 \xi \right) ) (A e^{i (\beta + 2n) \xi} + B e^{-i (\beta + 2n) \xi} ) = - \sum_{n = - \infty}^{\infty} C_{2n} (\beta + 2n)^2 (A e^{i (\beta + 2n) \xi} + B e^{-i (\beta + 2n) \xi} ) \\
	\implies \sum_{n = - \infty}^{\infty} C_{2n} (a_p - 2 q_x \cos (2 \xi) - (\beta + 2n)^2 ) (A e^{i (\beta + 2n) \xi} + B e^{-i (\beta + 2n) \xi} ) = 0 \\
	\implies \sum_{n = - \infty}^{\infty} C_{2n} \left( \frac{a_p - (\beta + 2n)^2}{q_x} - (e^{i 2 \xi} + e^{-i 2 \xi} ) \right) (A e^{i (\beta + 2n) \xi} + B e^{-i (\beta + 2n) \xi} ) = 0
\end{gather*}
where $\cos(2 \xi) = \frac{1}{2} (e^{i 2 \xi} + e^{-i 2 \xi})$ has been used. If we denote $D_{2n} = \frac{a_p - (\beta + 2n)^2}{q_x}$, then this further becomes:
\begin{gather*}
	\sum_{n = - \infty}^{\infty} C_{2n} \left( D_{2n} - e^{i 2 \xi} - e^{-i 2 \xi} \right) (A e^{i (\beta + 2n) \xi} + B e^{-i (\beta + 2n) \xi} ) = 0 \\
	\implies \sum_{n = - \infty}^{\infty} D_{2n} C_{2n} (A e^{i (\beta + 2n) \xi} + B e^{-i (\beta + 2n) \xi} ) - \sum_{n = - \infty}^{\infty} C_{2n} (A e^{i (\beta + 2n + 2) \xi} + B e^{-i (\beta + 2n + 2) \xi} ) \\ - \sum_{n = - \infty}^{\infty} C_{2n} (A e^{i (\beta + 2n - 2) \xi} + B e^{-i (\beta + 2n - 2) \xi} ) = 0
\end{gather*}
Changing the index in the second sum to $n \rightarrow 2n + 2$ and the index in the third sum to $n \rightarrow 2n - 2$, we achieve:
\begin{gather*}
	\sum_{n = - \infty}^{\infty} (D_{2n} C_{2n} - C_{2n - 2} - C_{2n + 2}) (A e^{i (\beta + 2n) \xi} + B e^{-i (\beta + 2n) \xi} ) = 0
\end{gather*}
Every Fourier component must be 0 in order to satisfy this condition, which leads to the following recurrence relation:
\begin{equation}
	D_{2n} C_{2n} - C_{2n + 2} - C_{2n - 2} = 0
\label{eq:phoebe}
\end{equation}
This connects the coefficients to the known parameters $a_p$ and $q_x$. It is most helpful to express the recurrence relation in continued fraction form, which can be done in either the increasing or decreasing (in $n$) direction.\par
\medskip
\noindent In the increasing direction, we generate:
\begin{align}
	\frac{C_{2n-2}}{C_{2n}} & = D_{2n} - \frac{C_{2n + 2}}{C_{2n}} \nonumber \\
	\frac{C_{2n}}{C_{2n - 2}} & = \frac{1}{D_{2n} - \frac{C_{2n + 2}}{C_{2n}}} \label{eq:ykb1} \\
	C_{2n} & = \frac{C_{2n - 2}}{D_{2n} - \frac{C_{2n + 2}}{C_{2n}}} \label{eq:yolo1} \\
	& = \frac{C_{2n - 2}}{D_{2n} - \frac{1}{D_{2n+2} - \frac{1}{\dots}}} \label{eq:CF1}
\end{align}
where the last line has been generated by iteratively substituting \eqref{eq:ykb1} into \eqref{eq:yolo1} $i$ times, with the index of \eqref{eq:ykb1} changed to $n \rightarrow n + i$. We can repeat the process in the decreasing direction, which generates:
\begin{align}
\frac{C_{2n + 2}}{C_{2n}} & = D_{2n} - \frac{C_{2n - 2}}{C_{2n}}	 \nonumber \\
\frac{C_{2n}}{C_{2n + 2}} & = \frac{1}{D_{2n} - \frac{C_{2n - 2}}{C_{2n}}} \label{eq:ykb2} \\
C_{2n} & = \frac{C_{2n + 2}}{D_{2n} - \frac{C_{2n - 2}}{C_{2n}}} \label{eq:yolo2} \\
& = \frac{C_{2n + 2}}{D_{2n} - \frac{1}{D_{2n-2} - \frac{1}{\dots}}} \label{eq:CF2}
\end{align}
Here, \eqref{eq:ykb2} has been iteratively substituted into \eqref{eq:yolo2} $i$ times, with the index of \eqref{eq:ykb2} changed to $n \rightarrow n - i$. If $C_{0} = 1$ is assumed without loss of generality, then \eqref{eq:CF1} can be used to find the $C_{2n}$ coefficients when $n$ is increasing from 0. Similarly, \eqref{eq:CF2} can be used to find the $C_{2n}$ coefficients when $n$ is decreasing from 0. Of course, we require the continued fraction expressions to be consistent for the same coefficient $C_{2n}$, which we enforce by deriving a condition on $\beta$. \par
\medskip
\noindent If we substitute $n = 1$ into \eqref{eq:CF1} and $n = 0$ into \eqref{eq:CF2}, then we generate the following two equations:
\begin{align}
C_2 & = \frac{C_0}{D_2 - \frac{1}{D_4 - \frac{1}{\dots}}} \\
C_0 & = \frac{C_2}{D_0 - \frac{1}{D_{-2} - \frac{1}{D_{-4} - \frac{1}{\dots}}}}
\end{align}
Rearranging these for $\frac{C_2}{C_0}$ and equating them, we achieve:
\begin{equation}
\frac{1}{D_2 - \frac{1}{D_4 - \frac{1}{\dots}}}	= D_0 - \frac{1}{D_{-2} - \frac{1}{D_{-4} - \frac{1}{\dots}}}
\end{equation}
$D_0 = \frac{a_p - \beta^2}{q_x}$ 
 





A continued fraction expression can also be generated for $\beta^2$:
\begin{equation}
\beta^2 = a_p - q_x \left( \frac{1}{D_0 - \frac{1}{D_2 - \frac{1}{...}}} + \frac{1}{D_0 - \frac{1}{D_{-2} - \frac{1}{...}}} \right) \label{eq:beta}
\end{equation}
We will consider the conditions necessary to reach the secular limit of the ion motion in \eqref{eq:motion} -- that is, the micromotion can be reasonably neglected. It is important to note that approximately secular motion cannot be achieved by simply setting $\omega_{RF} \rightarrow \infty$, as such a limit would cause the oscillatory component of the trap to do less and less in comparison to the static component. Earnshaw's theorem dictates that a static potential is unable to trap a charged ion, so $\omega_{RF} \rightarrow \infty$ would accordingly be unable confine the ion and no secular motion would be achievable. Rather, the secular limit can be achieved when the micromotion terms in \eqref{eq:motion} are not time-dependent, ie. the only non-zero micromotion coefficient is $C_0$ and all other terms for $n \neq 0$ vanish. In that case, the non-time-dependent terms can be absorbed into the constants $A$ and $B$. \par
\medskip
\noindent Via \eqref{eq:Cterms1} and \eqref{eq:Cterms2}, we achieve rapidly-vanishing $C_{2n}$ terms with increasing $n$ when the $D_{2n}$ terms also diverge rapidly with increasing $n$. In \eqref{eq:phoebe}, we see that the $D_{2n}$ terms will diverge quickly with $|n| \ge 1$ as long as $q_x$ and $a_p$ are both small. This indicates that for any given $\omega_{\text{sec}}$, our condition to achieve the approximately secular motion will involve small $q_x$ and $a_p$ values.

% Talk about the lowest-order approximation in section 2 from the paper! 

\end{document}